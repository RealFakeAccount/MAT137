\documentclass[12pt]{exam}
\usepackage{amsmath}
\usepackage{amssymb}
\usepackage{graphicx}
\usepackage{enumitem}
\usepackage{amsfonts}
\usepackage{amssymb}
\usepackage{ifthen}
\usepackage{geometry}
\noprintanswers

\usepackage{tikz}
\usetikzlibrary{shapes,backgrounds}

\usepackage{framed}

\addtolength{\textheight}{6cm}
\addtolength{\topmargin}{-2cm}
\addtolength{\textwidth}{3cm}
\addtolength{\oddsidemargin}{-1.5cm}
\addtolength{\evensidemargin}{-1.5cm}
\setlength\parindent{0pt}

\newcommand {\DS} [1] {${\displaystyle #1}$}
\newcommand{\vv}{\vspace{.1cm}}

\newcommand{\R}{\mathbb{R}}
\newcommand{\Q}{\mathbb{Q}}
\newcommand{\Z}{\mathbb{Z}}
\newcommand{\N}{\mathbb{N}}
\newcommand{\C}{\mathbb{C}}

\pagestyle{empty}

\graphicspath{ {./} }

%============================================
%137 COLOUR PALETTE
%============================================

\definecolor{137cp1}{RGB}{13, 33, 161}
\definecolor{137cp2}{RGB}{51, 161, 253}
\definecolor{137cp3}{RGB}{255, 67, 101}
\definecolor{137cp4}{RGB}{232, 144, 5}


%============================================
%HYPERLINKS
%============================================

\usepackage{hyperref}
\hypersetup{colorlinks}
\hypersetup{urlcolor=137cp3, linkcolor=137cp1}

%============================================
%Commands used only for this file
%============================================

\DeclareMathOperator{\arccot}{arccot}

%%%%%%%%%%%%%%%%%%%%%%%%%%%%%%%%%%%%%%%%%


\begin{document}

{\large
	\begin{center}
		{\bf MAT 137Y: Calculus with proofs}\\
		{\bf Test 3} \\
	\end{center}
}

\begin{enumerate}
	\item My example is 
	
	Let $n \in R$. 
	$\lim_{x \to \infty} x\frac{n}{x}$

	Using limit law to split them, the original equation equals to:  
	$$
	\lim_{x \to \infty} x \cdot  \lim_{x \to \infty} \frac{n}{x} = \infty \cdot 0
	$$


	This is an indeterminate forms because we can't draw conclusion from $0 \cdot \infty$

	However, by simplify it, we can get:

	$$
	\lim_{x \to \infty} x\frac{n}{x} = \lim_{x \to \infty} n = n \; \mbox{ (Property of limits)}
	$$
	
	By definition, n could be any real number. $\qquad \blacksquare$

% \newpage

	\item

	Draw from definition that 
	
	\begin{itemize}
		\item Let f be a bounded function on [a, b]
		\item Let $P = \{x_0,x_1,...,x_N\}$ be a partition of [a, b]
		\item For each $i = 1,...,N$, Let
		\begin{itemize}
			\item $m_i$ be the infimum of f on [$x_{i - 1}$, $x_i$]
			\item $M_i$ be the supremum of f on [$x_{i - 1}$, $x_i$]
			\item $\Delta x_i = x_i - x_{i - 1}$
		\end{itemize}
	\end{itemize}

	
	Then $L_P (f) = \sum_{i = 1}^{N} m_i \cdot \Delta x_i$ 

	Since $f$ is bounded on [a, b], it has a supremum on [a, b]
	We define $l$ be such supremum.

	From definition, we know that $m_i$ be the infimum of f on [$x_{i - 1}$, $x_i$]

	So, $\forall P. \forall m_i \; \mbox{defined by} \; P. m_i \leq l$.
	take this back to the definition of $L_P$:

	$$
	\forall P. L_P(f) \leq \sum_{i = 1}^{N}l \cdot \Delta x_i = \sum_{i = 1}^{N}l \cdot \sum_{i = 1}^{N} \Delta x_i = \sum_{i = 1}^{N}l \cdot (b - a)
	$$

	Which match the definition of supremum.

	So $\{L_P(f) | P$ is a partition of $[a, b]\}$ has a supremum. $\qquad \blacksquare$

\item 

we can rewrite the following equation as :

\begin{align*}
	& \lim_{x \to 0} \int_{0}^{x} \frac{(x - t)(f(t))}{x^2} dt \\
	= & \lim_{x \to 0} \frac{\int_{0}^{x} (x - t) f(t) dt}{x^2} \; \mbox{ (Property of integral)} \\
	= & \lim_{x \to 0} \frac{(x - x) f(x)}{2x} \; \mbox{(using L'Hôpital's rule) } \\
	= & 0 \quad \blacksquare
\end{align*}

\end{enumerate}

\end{document}

%%%%%%%%%%%%%%%%%%%%%%%%%%%%%%%%%%%%%%%
%%%%%%%%%%%%%%%%%%%%%%%%%%%%%%%%%%%%%%%
%%%%%%%%%%%%%%%%%%%%%%%%%%%%%%%%%%%%%%%

