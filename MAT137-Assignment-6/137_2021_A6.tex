\documentclass[12pt]{exam}
\usepackage{amsmath}
\usepackage{amssymb}
\usepackage{graphicx}
\usepackage{enumitem}
\usepackage{amsfonts}
\usepackage{amssymb}
\usepackage{ifthen}
\usepackage{geometry}
\noprintanswers

\usepackage{tikz}
\usetikzlibrary{shapes,backgrounds}

\usepackage{framed}

\addtolength{\textheight}{3.5cm}
\addtolength{\topmargin}{-0.8cm}
\addtolength{\textwidth}{2.5cm}
\addtolength{\oddsidemargin}{-1.5cm}
\addtolength{\evensidemargin}{-1.5cm}
\setlength\parindent{0pt}

\newcommand*\circled[1]{\tikz[baseline=(char.base)]{
    \node[shape=circle, draw, inner sep=1pt, 
        minimum height=12pt] (char) {#1};}}

\newcommand {\DS} [1] {${\displaystyle #1}$}
\newcommand{\vv}{\vspace{.1cm}}

\newcommand{\R}{\mathbb{R}}
\newcommand{\Q}{\mathbb{Q}}
\newcommand{\Z}{\mathbb{Z}}
\newcommand{\N}{\mathbb{N}}

\pagestyle{empty}


%============================================
%137 COLOUR PALETTE
%============================================

\definecolor{137cp1}{RGB}{13, 33, 161}
\definecolor{137cp2}{RGB}{51, 161, 253}
\definecolor{137cp3}{RGB}{255, 67, 101}
\definecolor{137cp4}{RGB}{232, 144, 5}

% to use colours easily
\newcommand{\azul}[1]{{\color{blue} #1}}
\newcommand{\rojo}[1]{{\color{red} #1}}
 
% box in red and blue in math and outside of math
\newcommand{\cajar}[1]{\boxed{\mbox{\rojo{ #1}}}}
\newcommand{\majar}[1]{\boxed{\rojo{ #1}}}
\newcommand{\cajab}[1]{\boxed{\mbox{\azul{ #1}}}}
\newcommand{\majab}[1]{\boxed{\azul{ #1}}}

%============================================
%HYPERLINKS
%============================================

\usepackage{hyperref}
\hypersetup{colorlinks}
\hypersetup{urlcolor=137cp3, linkcolor=137cp1}

%============================================
%Commands used only for this file
%============================================

\newcommand{\e}{\varepsilon}

%%%%%%%%%%%%%%%%%%%%%%%%%%%%%%%%%%%%%%%%%


\begin{document}

{\large
	\begin{center}
		{\bf MAT 137Y: Calculus with proofs}\\
		{\bf Assignment 6} \\
		{\bf Due on Thursday, January 28 by 11:59pm via Crowdmark}
	\end{center}
}

\vv

\begin{quotation}
{\bf Instructions:}
	\begin{itemize}
		\item	 You will need to submit your solutions electronically via Crowdmark.   \href{https://www.math.toronto.edu/~alfonso/137/PS/137_CM.html}{See MAT137 Crowdmark help page for instructions}.  Make sure you understand how to submit and that you try the system ahead of time.  If you leave it for the last minute and you run into technical problems, you will be late.  There are no extensions for any reason.
		\item You may submit individually or as a team of two students.  See the link above for more details.
		\item  You will need to submit your answer to each question separately.
		\item  This problem set is about Unit 7.
	\end{itemize}
\end{quotation}

\


The goal of this assignment is to prove the following result from the definition of integral:
	\begin{quotation}
	\noindent {\bf Theorem 1:}  
	Let $a<b$.  \; Let $f$ and $g$ be bounded functions on $[a,b]$.  Let $h = f+g$.
	\begin{itemize}
		\item  IF $f$ and $g$ are integrable on $[a,b]$
		\item  THEN $h$ is integrable on $[a,b]$ and
			$$
				\int_a^b h(x) dx \; = \;  \int_a^b f(x) dx \; + \; \int_a^b g(x) dx
			$$
	\end{itemize}
	\end{quotation}

We will break the proof into pieces and guide you through them.  In mathematical terms, you will be proving a few ``lemmas" before you prove Theorem 1.  For all the questions, let $a<b$ and let $f$ and $g$ bounded functions on $[a,b]$.  We won't repeat this every time.  {\bf Do not assume that any of the functions is integrable, unless specified}: many of the intermediate results hold for non-integrable functions as well.    In many of the questions, you will need to use the results of one or various previous questions in your proof.

\


\begin{enumerate}

\item  \begin{enumerate}
	\item It is NOT necessarily true that for every partition $P$ of $[a,b]$
		\begin{equation} \label{eq:Lfgh}
			L_{P}(f) \; + \; L_P(g) \; = \; L_P(h).
		\end{equation}
	Show it with a counterexample.
	
	\textbf{Solution:}
		No, it is \textbf{NOT} always true that
	$$
			L_{P}(f) \; + \; L_P(g) \; = \; L_P(h).
	$$
	I will give an counterexample:
	
	Let $f$, $g$ and $h=f+g$ be functions that are bounded on $[-1,1]$.
	
	Let $P=\{-1, 1\}$ represent a partition of $[-1,1]$
	
	Let $f$ be a function defined by:
	
	\[
	    f = 
	    \begin{cases}
	        4 & x \leq 0 \\
	        6 & x > 0
	    \end{cases}
	\]
	
	Let $g$ be a function defined by:
	
	\[
	    g = 
	    \begin{cases}
	        6 & x \leq 0 \\
	        4 & x > 0
	    \end{cases}
	\]
	
	h is defined by f + g, which is h = 10.
	
	From definition, we can calculate that $L_{P}(f) = 8, L_{P}(g) = 8$, while $L_{P}(h) = 20$
	
	We can find that 
	
	$$
	    L_{P}(f) \; + \; L_P(g) \; =  16 \neq \; L_P(h) = 20.\quad\blacksquare
	$$
	
	\newpage


	\item 	However, if we turn the equality into an inequality in \eqref{eq:Lfgh}, then it becomes true.  Which inequality?  Prove it.
	
	\textbf{Solution:}
	
	The inequality is $\leq$.
	
	We want to prove $L_{P}(f) \; + \; L_{P}(g) \; \leq \; L_{P}(h).$ always true. Let $f$, $g$ and $h=f+g$ be functions that are bounded on $[a,b]$.
	
	Let $P=\{x_0,x_1,...,x_n\}$ be a partition of $[a,b]$.
	For each i = 1,...,n,
	
	We denote $m_{i(f)}$, $m_{i(g)}$ and $m_{i(h)}$ as infimum of functions $f$, $g$ and $h$ on $[x_{i-1}, x_i]$ respectively. Also $\Delta x_i=x_i-x_{i-1}$.
	
	Then we can get that:
	\begin{align*}
	    L_{P}(f)&=\sum_{i=1}^n m_{i(f)}\Delta x_i\\
	    L_{P}(g)&=\sum_{i=1}^n m_{i(g)}\Delta x_i\\
	    L_{P}(h)&=\sum_{i=1}^n m_{i(h)}\Delta x_i
	\end{align*}
	Adding $L_{P}(f)$ and $L_{P}(g)$ together:
	$$
	    L_{P}(f)+L_{P}(g)=\sum_{i=1}^n m_{i(f)}\Delta x_i+\sum_{i=1}^n m_{i(g)}\Delta x_i=\sum_{i=1}^n (m_{i(f)}+m_{i(g)})\Delta x_i
	$$
	Then we left to show $m_{i(f)}+m_{i(g)}\leq m_{i(h)}$. For any constant $k\in[x_{i-1},x_i]$, since $m_{i(f)}$ and $m_{i(g)}$ are infimum on $[x_{i-1},x_i]$, then $m_{i(f)}<f(k)$ and $m_{i(g)}<g(k)$. As a result, $m_{i(f)}+m_{i(g)}\leq f(k)+g(k)=h(k)$. We know function $h$ also have a infimum of $m_{i(h)}\leq h(k)$. Take the greatest lower bound of $m_{i(h)}$ then $m_i{(h)}\geq m_{i(f)}+m_{i(g)}$. By this we can imply and conclude
	$$
	 L_{P}(f)+L_{P}(g)=\sum_{i=1}^n (m_{i(f)}+m_{i(g)})\Delta x_i\leq\sum_{i=1}^n m_{i(h)}\Delta x_i= L_{P}(h)\quad\blacksquare
	$$
	\end{enumerate}
	
\newpage

\item   {\bf [Do not submit]}  Prove that for every $\e>0$, there exists a partition $P$ of $[a,b]$ such that 
	$$
		\underline{I_a^b}(f) - \e \; < \;  L_P(f). 
	$$
	\emph{Note:}  This is a very, very short proof if you understand the definition of lower integral as supremum.  You may even have learned something similar in class.  You do not need to submit your answer to this question, but we want to make sure you think about it before trying the harder (and related) next question.

\item  Prove that for every $\e>0$, there exists a partition $P$ of $[a,b]$ such that
	$$
		\underline{I_a^b}(f) \, + \, \underline{I_a^b}(g) \, - \, \e \; < L_P(f) \, + \, L_P(g).
	$$
	\emph{Hint:}  This will feel a bit like those ``$\e$-$\delta$" proofs you learned in Unit 2.
	
	\textbf{Solution:}
	
	As proven in (2), $$\forall \e>0.\exists \mbox{ partition } P_1 \mbox{ of } [a,b]\mbox{ such that } \underline{I_a^b}(f) - \e \; < \;  L_{P_1}(f).\qquad\circled{1}$$
	
	Also, $$\forall \e>0.\exists \mbox{ partition } P_2 \mbox{ of } [a,b]\mbox{ such that } \underline{I_a^b}(g) - \e \; < \;  L_{P_2}(g).\qquad\circled{2}$$
	
	\textbf{Proof:}
	
	Let $\e>0$.
	
	We can use $\frac{\e}{3}$ in \circled{1} and \circled{2}, then we will have two partition $P_1$ and $P_2$ of $[a,b]$ satisfy:
	\begin{align*}
	    \underline{I_a^b}(f) - \frac{\e}{3} \; < \;  L_{P_1}(f).\qquad\circled{3}\\
	    \underline{I_a^b}(g) - \frac{\e}{3} \; < \;  L_{P_2}(g).\qquad\circled{4}
	\end{align*}
	Let $P=P_1\cup P_2$.
	Adding \circled{3} and \circled{4} together, we get
	$$
	    \underline{I_a^b}(f)+\underline{I_a^b}(g)-\frac{2\e}{3}<L_{P_1}(f)+L_{P_2}(g)
	$$
	Since $L_{P}(f)\geq L_{P_1}(f)$ and $L_{P}(g)\geq L_{P_2}(g)$ as $P$ is a refinement of $P_1$ and $P_2$ respectively, then
	$$
	    L_{P_1}(f)+L_{P_2}(g)\leq L_{P}(f)+L_{P}(g)
	$$
	Since $\e>0$, then
	$$
	    \underline{I_a^b}(f)+\underline{I_a^b}(g)-\e<\underline{I_a^b}(f)+\underline{I_a^b}(g)-\frac{2\e}{3}
	$$
	We can then conclude that:
	\begin{align*}
	    \underline{I_a^b}(f)+\underline{I_a^b}(g)-\e<\underline{I_a^b}(f)+\underline{I_a^b}(g)-\frac{2\e}{3}& < L_{P_1}(f)+L_{P_2}(g)\leq L_{P}(f)+L_{P}(g)\mbox{, which is }\\
	    \underline{I_a^b}(f)+\underline{I_a^b}(g)-\e&\leq L_{P}(f)+L_{P}(g)\qquad\blacksquare
	\end{align*}

\newpage


\item Prove that
	$$
		\underline{I_a^b}(f) \; + \; \underline{I_a^b}(g) \; \leq \;  \underline{I_a^b}(h).
	$$


	\emph{Note:}  If, at this moment, you think you have proven a strict inequality instead of a non-strict inequality, then your argument is probably wrong.
	
	\textbf{Solution:}
	
	Let $f$, $g$ and $h=f+g$ be functions that are bounded on $[a,b]$.
	
	Recall proven in (3), for every $\e>0$, there exists a partition $P$ of $[a,b]$ such that
	$$
		\underline{I_a^b}(f) \, + \, \underline{I_a^b}(g) \, - \, \e \; < L_P(f) \, + \, L_P(g).
	$$
	Take partition $P$ of $[a,b]$ satisfy inequality above.
	It can be turned into:
	\begin{align*}
	    \underline{I_a^b}(f)+\underline{I_a^b}(g)-L_P(f)-L_P(g)&<\e\\
	    \underline{I_a^b}(f)+\underline{I_a^b}(g)-L_P(f)-L_P(g)&\leq 0\qquad( \forall \e. \e>0)\\
	    \underline{I_a^b}(f)+\underline{I_a^b}(g)&\leq L_P(f)+L_P(g)\qquad\circled{5}
	\end{align*}
	Recall proven in (1b), for every partition $P$ of $[a,b]$,
	$$
	    L_{P}(f) \; + \; L_P(g) \; \leq \; L_P(h).\qquad\circled{6}
	$$
	By the definition of lower integral, for every partition $P$ of $[a,b]$, we know
	$$
	    \underline{I_a^b}(h)\geq L_{P}(h)\qquad\circled{7}
	$$
	Combining \circled{5}, \circled{6} and \circled{7} together:
	\begin{align*}
	    \underline{I_a^b}(f)+\underline{I_a^b}(g)\leq L_P(f)+L_P(g)&\leq L_P(h)\leq\underline{I_a^b}(h)\\
	    \underline{I_a^b}(f)+\underline{I_a^b}(g)&\leq \underline{I_a^b}(h)\qquad\blacksquare
	\end{align*}

	\newpage


\item  This question is irrelevant to the proof of Theorem 1, but it is also interesting.  Is it always true that 
	$$
		\underline{I_a^b}(f) \; + \; \underline{I_a^b}(g) \; = \; \underline{I_a^b}(h)  ?
	$$
	Prove it.
	..
	
	
	\textbf{Solution:}
	
	No, it is \textbf{NOT} always true that
	$$
		\underline{I_a^b}(f) \; + \; \underline{I_a^b}(g) \; = \; \underline{I_a^b}(h)
	$$
	I will give an counterexample:
	
	Let $f$, $g$ and $h=f+g$ be functions that are bounded on $[0,1]$.
	
	Let $P=\{x_0,x_1,...,x_n\}$ represent any partition of $[0,1]$
	
	Let $f$ be a function defined by:
	
	\[
	    f = 
	    \begin{cases}
	        4 & x \in \Q \\
	        6 & x \notin \Q
	    \end{cases}
	\]
	
	Let $g$ be a function defined by:
	
	\[
	    g = 
	    \begin{cases}
	        6 & x \in \Q \\
	        4 & x \notin \Q
	    \end{cases}
	\]
	Then in every sub-interval $[x_{i-1},x_i]$, there will contains a rational number or a irrational number. If a sub-interval contains a rational number then $\underline{I_a^b}(g)=4$. If a sub-interval contains irrational number then $\underline{I_a^b}(g)=4$. Since $h=f+g$, then it is a constant function that contains value $10$. So for any partition $P$ of $[0,1]$, the infimum of $h=f+g$ on each sub-interval is $10$. So, we can get $L_P(h)=10$ for any partition $P$ of $[0,1]$. We can then conclude $\underline{I_a^b}(h)=10$ which leads to $\underline{I_a^b}(f)+\underline{I_a^b}(g)=4+4=8\neq10=\underline{I_a^b}(h).\quad\blacksquare$
	
	\newpage


\item {\bf [Do not submit]}  Repeat the steps from the previous questions (with upper rather than lower sums and integrals) to prove that
	$$
		\overline{I_a^b}(h) \; \leq \; \overline{I_a^b}(f) \; + \; \overline{I_a^b}(g).
	$$

\item Prove Theorem 1.

\textbf{Proof:}

We have proven in (4):

$$
	\underline{I_a^b}(f) \; + \; \underline{I_a^b}(g) \; \leq \;  \underline{I_a^b}(h).
$$

We have proven in (6):

$$
    \overline{I_a^b}(h) \; \leq \; \overline{I_a^b}(f) \; + \; \overline{I_a^b}(g).
$$

Suppose $f, g$ is integrable on $[a,b]$, we can write:

$$
     \int_a^b f(x) dx \; = \underline{I_a^b}(f) \; = \overline{I_a^b}(f) \;
$$
and

$$
    \int_a^b g(x) dx \; = \underline{I_a^b}(g) \; = \overline{I_a^b}(g) \;
$$

Combining them, we get:

\begin{align*}
    \overline{I_a^b}(h) \; \leq \; \overline{I_a^b}(f) \; + \; \overline{I_a^b}(g) &= \underline{I_a^b}(f) \; + \; \underline{I_a^b}(g) \; \leq \;  \underline{I_a^b}(h) \circled{8} \\ 
    \overline{I_a^b}(h) &\leq \underline{I_a^b}(h) \circled{9}
\end{align*}

Also, we know that:

$$
    \overline{I_a^b}(h) = inf\{\mbox{upper sums of }h\}
$$

$$
    \underline{I_a^b}(h) = sup\{\mbox{lower sums of }h\}
$$

We can imply that $\overline{I_a^b}(h) \geq \underline{I_a^b}(h) \circled{10}$

Combining $\circled{9}$, $\circled{10}$, we can get that $\overline{I_a^b}(h) = \underline{I_a^b}(h)$

This means $h$ is integrable on bounded $[a,b]$.

Also, resuming from $\circled{8}$,

\begin{align*}
    \overline{I_a^b}(h) \; \leq \; \overline{I_a^b}(f) \; + \; \overline{I_a^b}(g) &= \underline{I_a^b}(f) \; + \; \underline{I_a^b}(g) \; \leq \;  \underline{I_a^b}(h) \\
    \overline{I_a^b}(h) \; = \; \overline{I_a^b}(f) \; + \; \overline{I_a^b}(g) &= \underline{I_a^b}(f) \; + \; \underline{I_a^b}(g) \; = \;  \underline{I_a^b}(h) \; \; (\mbox{since }\overline{I_a^b}(h) = \underline{I_a^b}(h))
\end{align*}

which implies
$$
    \int_a^b h(x) dx \; = \;  \int_a^b f(x) dx \; + \; \int_a^b g(x) dx \quad\blacksquare
$$
\end{enumerate}
\end{document}
%%%%%%%%%%%%%%%%%%%%%%%%%%%%%%%%
%%%%%%%%%%%%%%%%%%%%%%%%%%%%%%%%


	