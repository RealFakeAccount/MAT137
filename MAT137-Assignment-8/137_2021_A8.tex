\documentclass[12pt]{exam}
\usepackage{amsmath}
\usepackage{amssymb}
\usepackage{graphicx}
\usepackage{enumitem}
\usepackage{amsfonts}
\usepackage{amssymb}
\usepackage{ifthen}
\usepackage{geometry}
\noprintanswers

\usepackage{tikz}
\usetikzlibrary{shapes,backgrounds}

\usepackage{framed}

\addtolength{\textheight}{3.5cm}
\addtolength{\topmargin}{-0.8cm}
\addtolength{\textwidth}{2.5cm}
\addtolength{\oddsidemargin}{-1.5cm}
\addtolength{\evensidemargin}{-1.5cm}
\setlength\parindent{0pt}
\setlength{\parskip}{1em}

\newcommand {\DS} [1] {${\displaystyle #1}$}
\newcommand{\vv}{\vspace{.1cm}}

\newcommand{\R}{\mathbb{R}}
\newcommand{\Q}{\mathbb{Q}}
\newcommand{\Z}{\mathbb{Z}}
\newcommand{\N}{\mathbb{N}}

\pagestyle{empty}


%============================================
%137 COLOUR PALETTE
%============================================

\definecolor{137cp1}{RGB}{13, 33, 161}
\definecolor{137cp2}{RGB}{51, 161, 253}
\definecolor{137cp3}{RGB}{255, 67, 101}
\definecolor{137cp4}{RGB}{232, 144, 5}

% to use colours easily
\newcommand{\azul}[1]{{\color{blue} #1}}
\newcommand{\rojo}[1]{{\color{red} #1}}
 
% box in red and blue in math and outside of math
\newcommand{\cajar}[1]{\boxed{\mbox{\rojo{ #1}}}}
\newcommand{\majar}[1]{\boxed{\rojo{ #1}}}
\newcommand{\cajab}[1]{\boxed{\mbox{\azul{ #1}}}}
\newcommand{\majab}[1]{\boxed{\azul{ #1}}}

%============================================
%HYPERLINKS
%============================================

\usepackage{hyperref}
\hypersetup{colorlinks}
\hypersetup{urlcolor=137cp3, linkcolor=137cp1}

%============================================
%Commands used only for this file
%============================================

\newcommand{\e}{\varepsilon}

%%%%%%%%%%%%%%%%%%%%%%%%%%%%%%%%%%%%%%%%%


\begin{document}

{\large
	\begin{center}
		{\bf MAT 137Y: Calculus with proofs}\\
		{\bf Assignment 8} \\
		{\bf Due on Thursday, March 4 by 11:59pm via Crowdmark}
	\end{center}
}

\begin{quotation}
{\bf Instructions:}
	\begin{itemize}
		\item	 You will need to submit your solutions electronically via Crowdmark.   \href{https://www.math.toronto.edu/~alfonso/137/PS/137_CM.html}{See MAT137 Crowdmark help page for instructions}.  Make sure you understand how to submit and that you try the system ahead of time.  If you leave it for the last minute and you run into technical problems, you will be late.  There are no extensions for any reason.
		\item You may submit individually or as a team of two students.  See the link above for more details.
		\item  You will need to submit your answer to each question separately.
		\item  This problem set is about Unit 11.
	\end{itemize}
\end{quotation}

{\bf Notation:}  We will denote the set of positive integers by $\Z^+$.

\begin{enumerate}

\item Prove the following lemma:
	\begin{quotation} \noindent
	{\bf Lemma A.}
		 Let \DS{\left\{x_n\right\}_{n=1}^{\infty}} be a sequence of real numbers.  We define two new sequences \DS{\left\{E_n\right\}_{n=1}^{\infty}} and \DS{\left\{O_n\right\}_{n=1}^{\infty}} as:
			$$
			\begin{aligned}
				\forall n \in \Z^+, \; E_n&=x_{2n}, \\
				\forall n \in \Z^+, \; O_n&=x_{2n-1} 
			\end{aligned}
			$$
		\begin{itemize}
			\item IF the sequences \DS{\left\{E_n\right\}_{n=1}^{\infty}} and \DS{\left\{O_n\right\}_{n=1}^{\infty}} are both convergent to the {\bf \emph{same}} limit,
			\item THEN the sequence \DS{\left\{x_n\right\}_{n=1}^{\infty}} is also convergent.
		\end{itemize}
	\end{quotation}
	\emph{Suggestion:} Use the definition of limit.

\vv

\item Let \DS{\left\{a_n\right\}_{n=1}^{\infty}} be a \emph{decreasing} sequence of positive numbers with limit $0$.  I define a new sequence \DS{\left\{x_n\right\}_{n=1}^{\infty}} as follows:
	$$
	\begin{aligned}
		x_1 &= a_1 \\
		\forall n \in \Z^+, \quad x_{n+1} &= x_n + (-1)^n a_{n+1}
	\end{aligned}
	$$
	
	 Prove that the sequence \DS{\left\{x_n\right\}_{n=1}^{\infty}} satisfies the hypotheses of Lemma A, and hence is it convergent.

	This question is quite long and you will need to prove a few different things.  Before you start, make a strategy.  Decide what the various things you need to prove are, and in which order.  Begin the proof by writing a summary of the steps you are going to take.  Make sure your reader understands where in your proof you are at each moment.  Make your proof as easy to read as you would like it to be if you were reading it for the first time yourself.
		 
	 \emph{Suggestions:}  You do not need to write a single $\e$!  Use the theorems you have learned in Unit 11 instead.
	 Before you start, as rough work, write explicitly an equation for the first few $x$'s in terms of the first few $a$'s to make sure you understand the definition.  Draw the numbers $x_1$, $x_2$, $x_3$, $x_4$, $x_5$, $x_6$, $x_7$, and $x_8$ in a real line and order them.   You will notice that the sequences \DS{\left\{E_n\right\}_{n=1}^{\infty}}  and \DS{\left\{O_n\right\}_{n=1}^{\infty}} appear to satisfy certain properties which will be helpful in your proof.  Of course, anything you want to use in your proof needs to be proven first.
	
\vv

\item Let \DS{\left\{x_n\right\}_{n=1}^{\infty}} and \DS{\left\{y_n\right\}_{n=1}^{\infty}} be two sequences of positive numbers.  Assume that \DS{x_n << y_n}. 
	For each one of the following claims, decide whether they are always true, always false, or sometimes true and sometimes false (depending on the specific sequences).  Prove it.
	\begin{enumerate}
		\item \DS{x_n << \frac{x_n+y_n}{2}}
		\item \DS{\frac{x_n+y_n}{2} << y_n}
	\end{enumerate}

\end{enumerate}
\end{document}
%%%%%%%%%%%%%%%%%%%%%%%%%%%%%%%%
%%%%%%%%%%%%%%%%%%%%%%%%%%%%%%%%

