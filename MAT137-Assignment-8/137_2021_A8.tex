\documentclass[12pt]{exam}
\usepackage{amsmath}
\usepackage{amssymb}
\usepackage{graphicx}
\usepackage{enumitem}
\usepackage{amsfonts}
\usepackage{amssymb}
\usepackage{ifthen}
\usepackage{geometry}
\noprintanswers

\usepackage{tikz}
\usetikzlibrary{shapes,backgrounds}

\usepackage{framed}

\addtolength{\textheight}{3.5cm}
\addtolength{\topmargin}{-0.8cm}
\addtolength{\textwidth}{2.5cm}
\addtolength{\oddsidemargin}{-1.5cm}
\addtolength{\evensidemargin}{-1.5cm}
\setlength\parindent{0pt}
\setlength{\parskip}{1em}

\newcommand {\DS} [1] {${\displaystyle #1}$}
\newcommand{\vv}{\vspace{.1cm}}

\newcommand{\R}{\mathbb{R}}
\newcommand{\Q}{\mathbb{Q}}
\newcommand{\Z}{\mathbb{Z}}
\newcommand{\N}{\mathbb{N}}

\pagestyle{empty}


%============================================
%137 COLOUR PALETTE
%============================================

\definecolor{137cp1}{RGB}{13, 33, 161}
\definecolor{137cp2}{RGB}{51, 161, 253}
\definecolor{137cp3}{RGB}{255, 67, 101}
\definecolor{137cp4}{RGB}{232, 144, 5}

% to use colours easily
\newcommand{\azul}[1]{{\color{blue} #1}}
\newcommand{\rojo}[1]{{\color{red} #1}}
 
% box in red and blue in math and outside of math
\newcommand{\cajar}[1]{\boxed{\mbox{\rojo{ #1}}}}
\newcommand{\majar}[1]{\boxed{\rojo{ #1}}}
\newcommand{\cajab}[1]{\boxed{\mbox{\azul{ #1}}}}
\newcommand{\majab}[1]{\boxed{\azul{ #1}}}

%============================================
%HYPERLINKS
%============================================

\usepackage{hyperref}
\hypersetup{colorlinks}
\hypersetup{urlcolor=137cp3, linkcolor=137cp1}

%============================================
%Commands used only for this file
%============================================

\newcommand{\e}{\varepsilon}

%%%%%%%%%%%%%%%%%%%%%%%%%%%%%%%%%%%%%%%%%


\begin{document}

{\large
	\begin{center}
		{\bf MAT 137Y: Calculus with proofs}\\
		{\bf Assignment 8} \\
		{\bf Due on Thursday, March 4 by 11:59pm via Crowdmark}
	\end{center}
}

\begin{quotation}
{\bf Instructions:}
	\begin{itemize}
		\item	 You will need to submit your solutions electronically via Crowdmark.   \href{https://www.math.toronto.edu/~alfonso/137/PS/137_CM.html}{See MAT137 Crowdmark help page for instructions}.  Make sure you understand how to submit and that you try the system ahead of time.  If you leave it for the last minute and you run into technical problems, you will be late.  There are no extensions for any reason.
		\item You may submit individually or as a team of two students.  See the link above for more details.
		\item  You will need to submit your answer to each question separately.
		\item  This problem set is about Unit 11.
	\end{itemize}
\end{quotation}

{\bf Notation:}  We will denote the set of positive integers by $\Z^+$.

\begin{enumerate}

\item Prove the following lemma:
	\begin{quotation} \noindent
	{\bf Lemma A.}
		 Let \DS{\left\{x_n\right\}_{n=1}^{\infty}} be a sequence of real numbers.  We define two new sequences \DS{\left\{E_n\right\}_{n=1}^{\infty}} and \DS{\left\{O_n\right\}_{n=1}^{\infty}} as:
			$$
			\begin{aligned}
				\forall n \in \Z^+, \; E_n&=x_{2n}, \\
				\forall n \in \Z^+, \; O_n&=x_{2n-1} 
			\end{aligned}
			$$
		\begin{itemize}
			\item IF the sequences \DS{\left\{E_n\right\}_{n=1}^{\infty}} and \DS{\left\{O_n\right\}_{n=1}^{\infty}} are both convergent to the {\bf \emph{same}} limit,
			\item THEN the sequence \DS{\left\{x_n\right\}_{n=1}^{\infty}} is also convergent.
		\end{itemize}
	\end{quotation}
	\emph{Suggestion:} Use the definition of limit.

\vv

\emph{Proof:}

Since \DS{\left\{E_n\right\}_{n=1}^{\infty}} and \DS{\left\{O_n\right\}_{n=1}^{\infty}} is convergent to the same limit, then

There exists $L \in \R$ such that

$$
	\forall \epsilon > 0. \exists n_0 \in \Z^+. s.t. \forall n \in \Z^+. n \geq n_0 \implies | x_{2n} - L | < \epsilon
$$
$$
	\forall \epsilon > 0. \exists n_1 \in \Z^+. s.t. \forall n \in \Z^+. n \geq n_1 \implies | x_{2n-1} - L | < \epsilon
$$

Let $\epsilon >0$. Let's take $n_2 = max\{2n_0, 2n_1-1\}$.

We can get:
\begin{align*}
    \forall n\geq n_2. (\exists k\in\Z^+.n=2k \implies x_n=x_{2k}=E_k) \mbox{ AND } (\exists k\in\Z^+ n=2k-1 \implies x_n=x_{2k-1}=O_k)
\end{align*}
If $n=2k$, we know $n=2k\geq n_2\geq 2n_0$, which implies $k\geq n_0$ which satisfies limit definition then we can get $|x_{2k}-L|<\epsilon$ which is equivalent to $|x_n-L|<\epsilon$.

If $n=2k-1$, we know $n=2k-1\geq n_2\geq 2n_1-1$, which implies $k\geq n_1$ which satisfies limit definition then we can get $|x_{2k-1}-L|<\epsilon$ which is equivalent to $|x_n-L|<\epsilon$.

This equals to
$$
\forall \epsilon > 0. \exists n^2\in\Z^+.\forall n \in \Z^+. n \geq n_2 \implies | x_n - L | < \epsilon
$$
which means \DS{\left\{x_n\right\}_{n=1}^{\infty}} is convergent.

\newpage

\item Let \DS{\left\{a_n\right\}_{n=1}^{\infty}} be a \emph{decreasing} sequence of positive numbers with limit $0$.  I define a new sequence \DS{\left\{x_n\right\}_{n=1}^{\infty}} as follows:
	$$
	\begin{aligned}
		x_1 &= a_1 \\
		\forall n \in \Z^+, \quad x_{n+1} &= x_n + (-1)^n a_{n+1}
	\end{aligned}
	$$
	
	 Prove that the sequence \DS{\left\{x_n\right\}_{n=1}^{\infty}} satisfies the hypotheses of Lemma A, and hence is it convergent.

	This question is quite long and you will need to prove a few different things.  Before you start, make a strategy.  Decide what the various things you need to prove are, and in which order.  Begin the proof by writing a summary of the steps you are going to take.  Make sure your reader understands where in your proof you are at each moment.  Make your proof as easy to read as you would like it to be if you were reading it for the first time yourself.
		 
	 \emph{Suggestions:}  You do not need to write a single $\e$!  Use the theorems you have learned in Unit 11 instead.
	 Before you start, as rough work, write explicitly an equation for the first few $x$'s in terms of the first few $a$'s to make sure you understand the definition.  Draw the numbers $x_1$, $x_2$, $x_3$, $x_4$, $x_5$, $x_6$, $x_7$, and $x_8$ in a real line and order them.   You will notice that the sequences \DS{\left\{E_n\right\}_{n=1}^{\infty}}  and \DS{\left\{O_n\right\}_{n=1}^{\infty}} appear to satisfy certain properties which will be helpful in your proof.  Of course, anything you want to use in your proof needs to be proven first.
	
\vv

\emph{Proof:}

Steps:

\begin{itemize}
	\item First, we will proof that $\forall n\in\Z^+. x_n=\sum_{i=1}^n(-1)^{i+1}a_i$
	\item Then, we will proof that we will prove sequence satisfies the hypotheses of Lemma A.
	\item Finally, we can assume seque convergent because of Lemma A.
\end{itemize}
    

First, we want to prove by induction that
$$
    \forall n\in\Z^+. x_n=\sum_{i=1}^n(-1)^{i+1}a_i
$$
Base Case:

By definition of sequence \DS{\left\{x_n\right\}_{n=1}^{\infty}}, we know that
$$
    x_1 = a_1
$$
When $n=1$, we can substitute $n=1$ to get:
$$
    x_1 = \sum_{i=1}^1(-1)^{i+1}a_i = (-1)^2a_1=1\cdot a_1=a_1
$$
This satisfies our definition of sequence \DS{\left\{x_n\right\}_{n=1}^{\infty}}.

Induction Step:

By definition of sequence \DS{\left\{x_n\right\}_{n=1}^{\infty}}, we know that
$$
    \forall n \in \Z^+, x_{n+1} = x_n + (-1)^n a_{n+1}
$$
Let $n\in\Z^+.$ We assume $x_n=\sum_{i=1}^n(-1)^{i+1}a_i$ is true.

Then we can substitute $x_n$ to the definition of sequence to get:
\begin{align*}
    x_{n+1} &= x_n + (-1)^n a_{n+1}\quad(\mbox{Definition of sequence}\left\{x_n\right\}_{n=1}^{\infty})\\
    &= \sum_{i=1}^n(-1)^{i+1}a_i + (-1)^n a_{n+1}\quad(\mbox{By Induction Assumption})\\
    &= \sum_{i=1}^{n+1}(-1)^{i+1}a_i - (-1)^{(n+1)+1}a_{n+1} + (-1)^n a_{n+1}\quad(\sum_{i=1}^{n+1}(-1)^{i+1}a_i= \sum_{i=1}^n(-1)^{i+1}a_i +(-1)^{(n+1)+1}a_{n+1})\\
    &=\sum_{i=1}^{n+1}(-1)^{i+1}a_i - (-1)^{(n+1)+1}a_{n+1} + (-1)^{n} a_{n+1}\cdot1\\
    &=\sum_{i=1}^{n+1}(-1)^{i+1}a_i - (-1)^{(n+1)+1}a_{n+1} + (-1)^{n} a_{n+1}\cdot(-1)^2\qquad((-1)^2=1)\\
    &=\sum_{i=1}^{n+1}(-1)^{i+1}a_i - (-1)^{(n+1)+1}a_{n+1} + (-1)^{n+2} a_{n+1}\qquad((-1)^n\cdot(-1)^2=(-1)^{n+2})\\
    &=\sum_{i=1}^{n+1}(-1)^{i+1}a_i\qquad((-1)^{(n+1)+1}a_{n+1} + (-1)^{n+2}=(-1)^{n+2} a_{n+1})
\end{align*}
Then we have proven
$$
    \forall n\in\Z^+. x_n=\sum_{i=1}^n(-1)^{i+1}a_i
$$
Second, we will prove sequence satisfies the hypotheses of Lemma A, and hence it is convergent.
We know $\forall n\in\Z^+. 2n\geq n\mbox{ AND } 2n\in\Z^+$ and $\forall n\in\Z^+. 2n-1\geq n\mbox{ AND } 2n-1\in\Z^+$. By what we prove above, we know
\begin{align*}
    &\forall n \in \Z^+.\ x_{2n}=\sum_{i=1}^{2n}(-1)^{i+1}a_i\\
    &\forall n \in \Z^+.\ x_{2n-1}=\sum_{i=1}^{2n-1}(-1)^{i+1}a_i
\end{align*}

We define two new sequences $\{E_n\}_{n=1}^{\infty}$ and $\{O_n\}_{n=1}^{\infty}$ as:
\begin{align*}
	\forall n \in \Z^+, \; E_n&=x_{2n}, \\
	\forall n \in \Z^+, \; O_n&=x_{2n-1} \\
\end{align*}

Since \DS{\left\{a_n\right\}_{n=1}^{\infty}} be a decreasing sequence of positive numbers with limit $0$. We know that $a_n>a_{n+1}$ since it is decreasing then $a_{n+1}-a_{n}<0\mbox{ AND }a_{n}-a_{n+1}>0$. Then we will write $E_n$ and $O_n$ explicitly:
\begin{align*}
    E_n&=x_{2n}\\
    &=a_1-a_2+a_3-a_4+\cdots+a_{2n-1}-a_{2n}\\\
    &=a_1+(a_3-a_2)+\cdots+(a_{2n-1}-a_{2n-2})-a_{2n}\qquad(a_{n+1}-a_{n}<0)\\
    &<a_1 \qquad(a_{2n}>0)(\mbox{Bounded above})\\
    O_n&=x_{2n-1}\\
    &=a_1-a_2+a_3-a_4+\cdots+a_{2n-1}\\\
    &=(a_1-a_2)+\cdots+(a_{2n-3}-a_{2n-2})+a_{2n-1}\qquad(a_{n}-a_{n+1}>0)\\
    &>0 \qquad(a_{2n-1}>0)(\mbox{Bounded below})
\end{align*}

Then we know that $E_{n+1}-E_{n}=a_{2n+1}-a_{2n+2}>0$ since $\{a_n\}_{n=1}^{\infty}$ is decreasing, so sequence $\{E_n\}_{n=1}^{\infty}$ is increasing and bounded above. By theorem, sequence $\{E_n\}_{n=1}^{\infty}$ is convergent and have a limit.

We also know that $O_{n+1}-O_{n}=a_{2n+1}-a_{2n}<0$ since $\{a_n\}_{n=1}^{\infty}$ is decreasing, so sequence $\{O_n\}_{n=1}^{\infty}$ is decreasing and bounded below. By theorem, sequence $\{O_n\}_{n=1}^{\infty}$ is convergent and have a limit.

By definition of $\{E_n\}_{n=1}^{\infty}$ and $\{O_n\}_{n=1}^{\infty}$ , $E_n-O_n=-a_{2n}$. Since $\{a_n\}_{n=1}^{\infty}$ is with limit 0,we can write $\lim_{n\to\infty}a_n=0$. Then this can imply $\lim_{n\to\infty}-a_{2n}=0$. Then we know that the limits of $\{E_n\}_{n=1}^{\infty}$ and $\{O_n\}_{n=1}^{\infty}$ are the same.

We have proven the assumption of Lemma A. By Lemma A, we know that sequence \DS{\left\{x_n\right\}_{n=1}^{\infty}} is also convergent. \qquad$\blacksquare$

\newpage

\item Let \DS{\left\{x_n\right\}_{n=1}^{\infty}} and \DS{\left\{y_n\right\}_{n=1}^{\infty}} be two sequences of positive numbers.  Assume that \DS{x_n << y_n}. 
	For each one of the following claims, decide whether they are always true, always false, or sometimes true and sometimes false (depending on the specific sequences).  Prove it.
	\begin{enumerate}
		\item \DS{x_n << \frac{x_n+y_n}{2}}
		
		\emph{Proof:}
		
		We have assume $x_n << y_n$. By definition of $<<$, it means $\lim_{n\to\infty}\frac{x_n}{y_n}=0$.
		
		Since \DS{\left\{x_n\right\}_{n=1}^{\infty}} and \DS{\left\{y_n\right\}_{n=1}^{\infty}} are two sequences of positive numbers, then $x_n>0$ and $y_n>0$. We can then get $\frac{x_n}{x_n+y_n}>0$ and implies $\frac{x_n}{x_n+y_n}\geq0$. We also know that $\frac{x_n}{y_n}>\frac{x_n}{x_n+y_n}$ and implies $\frac{x_n}{y_n}\geq\frac{x_n}{x_n+y_n}$.
		
		Since $\lim_{n\to\infty}0=0$ and $\lim_{n\to\infty}\frac{x_n}{y_n}=0$, also $0\leq\frac{x_n}{x_n+y_n}\leq\frac{x_n}{y_n}$, by Squeeze Theorem, $\lim_{n\to\infty}\frac{x_n}{(x_n+y_n)}=0$
		
		We want to calculate the limit $\lim_{n\to\infty}\frac{x_n}{\frac{x_n+y_n}{2}}$:
		\begin{align*}
		    \lim_{n\to\infty}\frac{x_n}{\frac{x_n+y_n}{2}}&=\lim_{n\to\infty}\frac{x_n}{\frac{1}{2}(x_n+y_n)}\\
		    &=\lim_{n\to\infty}2\cdot\frac{x_n}{(x_n+y_n)}\\
		    &=2\cdot\lim_{n\to\infty}\frac{x_n}{(x_n+y_n)}\qquad(\mbox{Limit Law})\\
		    &=2\cdot0\qquad(\lim_{n\to\infty}\frac{x_n}{(x_n+y_n)}=0)\\
		    &=0
		\end{align*}
		Then we have proven that \DS{x_n << \frac{x_n+y_n}{2}} is always true.\qquad$\blacksquare$
		\newpage
		
		\item \DS{\frac{x_n+y_n}{2} << y_n}
		
		\emph{Proof:}
		
		We have assume $x_n << y_n$. By definition of $<<$, it means $\lim_{n\to\infty}\frac{x_n}{y_n}=0$.
		
		We know that $\lim_{n\to\infty}\frac{1}{2}=\frac{1}{2}$. Then by limit law, $\lim_{n\to\infty}\frac{1}{2}\cdot\frac{x_n}{y_n}=\frac{1}{2}\cdot0=0$
		
		We want to calculate the limit $\lim_{n\to\infty}\frac{\frac{x_n+y_n}{2}}{y_n}$:
		\begin{align*}
		    \lim_{n\to\infty}\frac{\frac{x_n+y_n}{2}}{y_n}&=\lim_{n\to\infty}\frac{\frac{1}{2}(x_n+y_n)}{y_n}\\
		    &=\lim_{n\to\infty}\frac{1}{2}\cdot\frac{(x_n+y_n)}{y_n}\\
		    &=\lim_{n\to\infty}\frac{1}{2}\cdot[1+\frac{x_n}{y_n}]\\
		    &=\lim_{n\to\infty}\frac{1}{2}+\lim_{n\to\infty}\frac{1}{2}\cdot\frac{x_n}{y_n}\\
		    &=\frac{1}{2}+0\qquad(\mbox{Limit Law})\\
		    &=\frac{1}{2}
		\end{align*}
		Then we have proven that \DS{\frac{x_n+y_n}{2} << y_n} is always false\qquad$\blacksquare$
	\end{enumerate}

\end{enumerate}
\end{document}
%%%%%%%%%%%%%%%%%%%%%%%%%%%%%%%%
%%%%%%%%%%%%%%%%%%%%%%%%%%%%%%%%

