\documentclass[12pt]{exam}
\usepackage{amsmath}
\usepackage{amssymb}
\usepackage{graphicx}
\usepackage{enumitem}
\usepackage{amsfonts}
\usepackage{amssymb}
\usepackage{ifthen}
\usepackage{geometry}
\noprintanswers

\usepackage{tikz}
\usetikzlibrary{shapes,backgrounds}
\newcommand*\circled[1]{\tikz[baseline=(char.base)]{
    \node[shape=circle, draw, inner sep=1pt, 
        minimum height=12pt] (char) {#1};}}

\usepackage{framed}

\addtolength{\textheight}{4.5cm}
\addtolength{\topmargin}{-1.3cm}
\addtolength{\textwidth}{3.5cm}
\addtolength{\oddsidemargin}{-2cm}
\addtolength{\evensidemargin}{-2cm}
\setlength\parindent{0pt}

\newcommand {\DS} [1] {${\displaystyle #1}$}
\newcommand{\vv}{\vspace{.1cm}}

\newcommand{\R}{\mathbb{R}}
\newcommand{\Q}{\mathbb{Q}}
\newcommand{\Z}{\mathbb{Z}}
\newcommand{\N}{\mathbb{N}}

\pagestyle{empty}


%============================================
%137 COLOUR PALETTE
%============================================

\definecolor{137cp1}{RGB}{13, 33, 161}
\definecolor{137cp2}{RGB}{51, 161, 253}
\definecolor{137cp3}{RGB}{255, 67, 101}
\definecolor{137cp4}{RGB}{232, 144, 5}

% to use colours easily
\newcommand{\azul}[1]{{\color{blue} #1}}
\newcommand{\rojo}[1]{{\color{red} #1}}
 
% box in red and blue in math and outside of math
\newcommand{\cajar}[1]{\boxed{\mbox{\rojo{ #1}}}}
\newcommand{\majar}[1]{\boxed{\rojo{ #1}}}
\newcommand{\cajab}[1]{\boxed{\mbox{\azul{ #1}}}}
\newcommand{\majab}[1]{\boxed{\azul{ #1}}}

%============================================
%HYPERLINKS
%============================================

\usepackage{hyperref}
\hypersetup{colorlinks}
\hypersetup{urlcolor=137cp3, linkcolor=137cp1}

%============================================
%Commands used only for this file
%============================================


%%%%%%%%%%%%%%%%%%%%%%%%%%%%%%%%%%%%%%%%%


\begin{document}

{\large
	\begin{center}
		{\bf MAT 137Y: Calculus with proofs}\\
		{\bf Assignment 5} \\
		{\bf Due on Sunday, December 20 by 11:59pm via Crowdmark}
	\end{center}
}

\vv

\begin{quotation}
{\bf Instructions:}
	\begin{itemize}
		\item	 You will need to submit your solutions electronically via Crowdmark.   \href{https://www.math.toronto.edu/~alfonso/137/PS/137_CM.html}{See MAT137 Crowdmark help page for instructions}.  Make sure you understand how to submit and that you try the system ahead of time.  If you leave it for the last minute and you run into technical problems, you will be late.  There are no extensions for any reason.
		\item You may submit individually or as a team of two students.  See the link above for more details.
		\item  You will need to submit your answer to each question separately.
		\item  This problem set is about Unit 6.
	\end{itemize}
\end{quotation}
\vv

\begin{enumerate}

\item Every morning Neo packs his backpack and walks a distance $L$ through a straight path in the forest from his home ($A$) to the unicorn sanctuary ($B$).  One day he discovers someone has built an electric fence in the exact middle of his daily path (the dashed, red line in the picture):
\begin{center}
\begin{tikzpicture}
	\draw (0,0) to (6,0);
	\draw[red, very thick, dashed] (3,-1) to (3,1);
	\draw[fill] (0,0) circle [radius=.1];
	\draw[fill] (6,0) circle [radius=.1];
	\node[xshift=-10pt] at (0,0) {$A$} ;
	\node[xshift= 10pt] at (6,0) {$B$} ;
\end{tikzpicture}
\end{center}
The fence has length $2b$ --  it extends for a distance $b$ on each side of the path -- and is perpendicular to the path.  After a few days, Neo notices that the electric fence is turned on only half the time, but he does not know if the fence is on or off on any given day until he walks up to it and throws his cat at the fence to test it.  If the fence is off, he can just quickly climb over it.  Otherwise, he has to walk around it.  He devises a plan: he will walk straight from his home to some point $P$ in the fence; then, he will walk around it or climb over it depending on whether the fence is on or off.    Which point $P$ should he choose in order to minimize the \emph{average} length of his trip?


\vv

\item  
	\begin{enumerate}
				
		\item  Let $f$ be a function with domain $\R$.  Assume $f$ has derivatives of every order.  Find all possible real numbers $A, B, C \in \R$ such that
			\begin{equation} \label{eq:ABC}
				\lim_{x \to 0} \frac{f(x) - \left[ Ax^2 + B x + C \right]}{x^2} = 0.
			\end{equation}	
			
			\emph{Note:} In your answer, $A$, $B$ and $C$ will depend on values of $f$ and its derivatives.  We are asking for \emph{all} possible answers.  We want you to prove that your choices of $A$, $B$, and $C$ satisfy \eqref{eq:ABC}, and that there are no other choices that satisfy \eqref{eq:ABC}.

			\emph{proof:}
			
			We want $\lim_{x \to 0} \frac{f(x) - (Ax^2 + Bx + C)}{x^2} = 0 \circled{1}$. 

			Since $\lim_{x \to 0} x^2 =0$, the only form possible is indeterminate form ($\lim_{x \to 0} f(x) - (Ax^2 + Bx + C) = 0$).
			Otherwise, this limit does not exist.

			Since $f$ has derivatives of every order, we can proof it is valid to use L'Hôpital's Rule. 
			Using L'Hôpital's Rule, we get:

			$$
				\lim_{x \to 0} \frac{f'(x) - (2Ax + B)}{2x} = 0 \circled{2}
			$$

			Same as above, $\lim_{x \to 0} 2x =0$, the only form possible is indeterminate form ($\lim_{x \to 0} f'(x) - (2Ax + B) = 0$).
			Otherwise, this limit does not exist. we can use L'Hôpital's Rule and get:

			$$
				\lim_{x \to 0} \frac{f''(x) - 2A}{2} = 0 \circled{3}
			$$

			Since $f$ has derivatives of every order, f is continuous on every order. 
			Thus, $\lim_{x \to 0} f''(x) = f''(0)$

			We can and only can take $A = \frac{f''(0)}{2}$ to satisfy the equation $\circled{3}$.

			Take the A back to $\circled{2}$, we get:

			$$
				\lim_{x \to 0} \frac{f'(x) - (2Ax + B)}{2x} = 0
			$$

			We have proven that $\lim_{x \to 0} f'(x) - (2Ax + B) = 0$. 
			Using limit laws, we can get $\lim_{x \to 0} f'(x) - \lim_{x \to 0} 2Ax - \lim_{x \to 0}B = 0$.

			Same as above, $\lim_{x \to 0} f'(x) = f'(0)$. Since $\lim_{x \to 0} 2Ax = 0$, 
			we can and only can take $B = f'(0)$.

			We have proven that $\lim_{x \to 0} f(x) - (Ax^2 + Bx + C) = 0$.
			Using limit laws, we can get $\lim_{x \to 0} f(x) - \lim_{x \to 0}Ax^2 - \lim_{x \to 0}Bx - \lim_{x \to 0}C = 0$.

			Same as above, $\lim_{x \to 0} f(x) = f(0)$. Since $\lim_{x \to 0}Ax^2 = 0$ and $\lim_{x \to 0}Bx = 0$, 
			we can and only can take $C = f(0)$.

			We have shown that $A = \frac{f''(0)}{2}, B = f'(0), C = f(0)$ can satisfy (1), and that there are no other choice that satisfy (1).
			% \begin{align*}
			% 	&\lim_{x \to 0} \frac{f(x) - (Ax^2 + Bx + C)}{x^2} \\
			% 	=&\lim_{x \to 0} \frac{f(x) - C - (Ax^2 + Bx)}{x^2} \\
			% 	=&\lim_{x \to 0} \frac{f(x) - C}{x^2} - \lim_{x \to 0} \frac{Ax^2}{x^2} + \lim_{x \to 0} \frac{Bx}{x^2} \\
			% \end{align*}

			% Thus we can conclude that
			% %%%%%%%%%%%%%%%%%%%%%%%%%%%
			% %%need more details here?%%
			% %%%%%%%%%%%%%%%%%%%%%%%%%%%
			% $$
			% 	\lim_{x \to 0} \frac{Ax^2}{x^2} =  A\lim_{x \to 0} \frac{x^2}{x^2} = 0
			% $$
			% % $$
			% % 	\lim_{x \to 0} \frac{Bx}{x^2} = B\lim_{x \to 0} \frac{x}{x^2} = 0
			% % $$

			% \begin{align*}
			% 	\lim_{x \to 0} f'(x) -2Ax-B &= 0\\
			% 	\lim_{x \to 0} \frac{f'(x)}{2x} - \lim_{x \to 0} \frac{f''(0)x}{2x} - \lim_{x \to 0} \frac{B}{2x} &= 0
			% \end{align*}


			% $\lim_{x \to 0} \frac{Ax^2}{x^2}$ %and $\lim_{x \to 0} \frac{Bx}{x^2}$ 
			% always equal to 0 for all $A, B \in \R$.

			% Since we want the original equation equals to 0, the equation can be rewritten as:

			% \begin{align*}
			% 	0=&\lim_{x \to 0} \frac{f(x) - C}{x^2} - \lim_{x \to 0} \frac{Ax^2}{x^2} + \lim_{x \to 0} \frac{Bx}{x^2}\\
			% 	0=&\lim_{x \to 0} \frac{f(x) - C}{x^2} - \lim_{x \to 0} \frac{Bx}{x^2} + 0\\
			% 	%0=&\lim_{x \to 0} \frac{f(x)}{x^2} - \lim_{x \to 0} \frac{C}{x^2} \\
			% \end{align*}
			
			% Because $f$ has derivatives of every order, $\lim_{x \to 0} f(x) = f(0)$. 
			% Thus we can get $C=f(0)$.

			

		\item  Let $f$ be a function with domain $\R$.  Assume $f$ has derivatives of every order.   Let $N$ be a positive integer.  Find a polynomial \DS{P_N} such that
			$$
				\lim_{x \to 0} \frac{f(x) - P_N(x)}{x^N} = 0
			$$
			\emph{Suggestion:} You may want to do some rough work until you can form a conjecture.    Do not submit the rough work.  To prove your conjecture, use induction.

			\emph{proof:}

			$$
			P_N = (\sum_{k = 1}^{N}\frac{f^k(0)}{k!}x^k) + f(0)
			$$

			I'll proof it using induction.

			\textbf{base step}: Assume N = 1.
			WTS:
			$$
				\lim_{x \to 0} \frac{f(x) - P_1(x)}{x} = 0
			$$

			We can write left hand of the equation as:

			\begin{align*}
				&\lim_{x \to 0} \frac{f(x) - f'(0)x - f(0)}{x} \\
				=& \lim_{x \to 0} \frac{f(x) - f(0) - f'(0)x}{x}
			\end{align*}

			Since $\lim_{x \to 0}{f(x) - f(0)} = 0, \lim_{x \to 0}{f'(0)x} = 0$, and  $f$ has derivatives of every order,
			we could use L'H\^{o}pital's Rule:

			$$
				\lim_{x \to 0} \frac{f(x) - f(0) - f'(0)x}{x}
				= \lim_{x \to 0} f'(x) - f'(0)
			$$

			Since $f$ has derivatives of every order, f is continuous on every order. 
			Thus, $\lim_{x \to 0} f'(x) = f'(0)$. 
			Take this back to the equation, we can get
			$$
				\lim_{x \to 0} f'(x) - f'(0) = 0
			$$
			which means 
			$$
				\lim_{x \to 0} \frac{f(x) - P_1(x)}{x} = 0
			$$

			\textbf{Induction step:}
			
			Assume $P_N$ satisfy $$\lim_{x \to 0} \frac{f(x) - P_N(x)}{x^N} = 0$$

			\emph{WTS:} $$\lim_{x \to 0} \frac{f(x) - P_{N+1}(x)}{x^{N+1}} = 0$$ 
			
			We can expand $P_{N+1}$, so the left hand side of the equation can be written as:

			\begin{align*}
				&\lim_{x \to 0} \frac{f(x) - \sum_{k = 1}^{N + 1} \frac{f^k(0)x^k}{k!} - f(0)}{x^{N + 1}} \\
				=& \lim_{x \to 0} \frac{f(x) - \sum_{k = 1}^{N} \frac{f^k(0)x^k}{k!} - f(0) - \frac{f^{N + 1}(0)x^{N + 1}}{(N + 1)!}}{x^{N + 1}} \\
			\end{align*}

			We know that
			\begin{itemize}
				\item $\lim_{x \to 0} f(x) = 0$
				\item $\lim_{x \to 0} \sum_{k = 1}^{N} \frac{f^k(0)x^k}{k!} = 0$
				\item $\lim_{x \to 0} f(0) = 0$
				\item $\lim_{x \to 0} \frac{f^{N + 1}(0)x^{N + 1}}{(N + 1)!} = 0$
			\end{itemize}

			And we can use L'H\^{o}pital's Rule $N + 1$ times.
			The highest power of $\sum_{k = 1}^{N} \frac{f^k(0)x^k}{k!}$ is $N$, and $N < N + 1$.
			So after $N + 1$ times of calculating derivatives, $\sum_{k = 1}^{N} \frac{f^k(0)x^k}{k!}$ becomes 0.
			$f(0)$ becomes 0 in the first calculating.

			$\lim_{x \to 0} \frac{f^{N + 1}(0)x^{N + 1}}{(N + 1)!} = 0$ after $N + 1$ times of calculating derivatives, 
			is $f^{N + 1}(0)$.

			the equation, after $N + 1$ times of calculating derivatives, is:

			$$
				\lim_{x \to 0} \frac{f^{N + 1}(x) - f^{N + 1}(0)}{(N + 1)!}
			$$

			Since $f$ has derivatives of every order, f is continuous on every order. 
			Thus, $\lim_{x \to 0} f^{N + 1}(x) = f^{N + 1}(0)$. 

			This means $\lim_{x \to 0} f^{N + 1}(x) - f^{N + 1}(0) = 0$, which indicates that 
			$$\lim_{x \to 0} \frac{f(x) - P_{N+1}(x)}{x^{N+1}} = 0$$
			

		\item  Using your new result, find polynomials $P$  and $Q$ such that
			$$
				\lim_{x \to 0} \frac{e^x - P(x)}{x^6} = 0, \quad \quad \lim_{x \to 0} \frac{\sin x - Q(x)}{x^{11}} = 0.
			$$
	\end{enumerate}

\vv

\emph{Proof:}

\vv

We know that derivative of $e^x$ is $e^x$ which is as same as itself. Then we get the information that $f(x)=e^x$ has derivatives of every order.

For the first equation $$\lim_{x\to0}\frac{e^x-P(x)}{x^6}=0$$
We take $N=6$ and we can then apply the formula into $$\lim_{x \to 0} \frac{f(x) - P_N(x)}{x^N} = 0$$ By using
$$P_N = (\sum_{k = 1}^{N}\frac{f^k(0)}{k!}x^k) + f(0).$$ And we get and expand this polynomial:
\begin{align*}
    P(x)=P_6 &= (\sum_{k = 1}^{6}\frac{f^k(0)}{k!}x^k) + f(0).\\
    &=\frac{f^6(0)}{6!}x^6+\frac{f^5(0)}{5!}x^5+\frac{f^4(0)}{4!}x^4+\frac{f^3(0)}{3!}x^3+\frac{f^2(0)}{2!}x^2+\frac{f'(0)}{1!}x+f(0)\quad()\\
    &=\frac{1}{6!}x^6+\frac{1}{5!}x^5+\frac{1}{4!}x^4+\frac{1}{3!}x^3+\frac{1}{2!}x^2+\frac{1}{1!}x+1\quad(\frac{d}{dx}e^x=e^x\mbox{ and }e^0=1)\\
    &=\frac{x^6}{6!}+\frac{x^5}{5!}+\frac{x^4}{4!}+\frac{x^3}{3!}+\frac{x^2}{2!}+x+1
\end{align*}

We know that:
\begin{align*}
    &\frac{d}{dx}\sin{x}=\cos{x}\\
    &\frac{d}{dx}\cos{x}=-\sin{x}\\
    &\frac{d}{dx}-\sin{x}=-\cos{x}\\
    &\frac{d}{dx}-\cos{x}=\sin{x}
\end{align*}
Then we get the information that $f(x)=\sin{x}$ has derivatives of every order and $f^N(x)=f^{N+4}(x)$.

For the second equation $$\lim_{x \to 0} \frac{\sin x - Q(x)}{x^{11}} = 0$$
We take $N=11$ and we can then apply the formula into $$\lim_{x \to 0} \frac{f(x) - P_N(x)}{x^N} = 0$$ By using
$$P_N = (\sum_{k = 1}^{N}\frac{f^k(0)}{k!}x^k) + f(0).$$ And we get and expand this polynomial:
\begin{align*}
    Q(x)=P_{11} =& (\sum_{k = 1}^{11}\frac{f^k(0)}{k!}x^k) + f(0).\\
    =&\frac{f^{11}(0)}{11!}x^{11}+\frac{f^{10}(0)}{10!}x^{10}+\frac{f^9(0)}{9!}x^9+\frac{f^8(0)}{8!}x^8+\frac{f^7(0)}{7!}x^7+\frac{f^6(0)}{6!}x^6+\frac{f^5(0)}{5!}x^5+\\
    &\frac{f^4(0)}{4!}x^4+\frac{f^3(0)}{3!}x^3+\frac{f^2(0)}{2!}x^2+\frac{f^1(0)}{1!}x+f(0)\quad(\mbox{Expand Sum Formula})\\
    =&\frac{-\cos{0}}{11!}x^{11}+\frac{-\sin{0}}{10!}x^{10}+\frac{\cos{0}}{9!}x^9+\frac{\sin{0}}{8!}x^8+\frac{-\cos{0}}{7!}x^7+\frac{-\sin{0}}{6!}x^6+\frac{\cos{0}}{5!}x^5+\\
    &\frac{\sin{0}}{4!}x^4+\frac{-\cos{0}}{3!}x^3+\frac{-\sin{0}}{2!}x^2+\frac{\cos{0}}{1!}x+f(0)\quad(f^N(x)=f^{N+4}(x))\\
    =&\frac{-1}{11!}x^{11}+\frac{1}{9!}x^9+\frac{-1}{7!}x^7+\frac{1}{5!}x^5+\frac{-1}{3!}x^3+\frac{1}{1!}x\quad(\cos{0}=1\mbox{ and }\sin{0}=0)\\
    =&-\frac{x^{11}}{11!}+\frac{x^9}{9!}-\frac{x^7}{7!}+\frac{x^5}{5!}-\frac{x^3}{3!}+x
\end{align*}

We have found polynomials $P$ and $Q$ by our result from Q2b as needed. $\qquad\blacksquare$

\newpage

\vv

\item  In Video 6.13, you learned about various geometrical notions that we could have used to define concavity.  Here is yet another one.

 Let $f$ be a function defined on an interval $I$.  Given two points $P$ and $Q$ on the graph of $f$, we will call $m_{P,Q}$ the slope of the line going through $P$ and $Q$.  We say that the function $f$ is ``cave up" on $I$ when for every 3 different points $P$, $Q$, and $R$ on the graph of $f$, if $P$ is to the left of $Q$, and $Q$ is to the left of $R$, then $m_{P,Q} < m_{Q,R}$.  Sketch a graph and make sure you understand this definition geometrically  before continuing.
 
 Assume $f$ is differentiable on $I$. 
		Prove that IF $f$ is concave up on $I$, THEN $f$ is cave up on $I$.


\emph{Hint:}  Use MVT.

\emph{Note:}  It is also possible to prove that cave up implies concave up, but we will skip it for now.  In fact, all of the different versions of concavity you have learned are equivalent for differentiable functions.	

\vv

\item  Let's recall the definition of horizontal/slant asymptote.  Let $f$ be a function defined at least on an interval $(c,\infty)$ for some $c \in \R$.
We say that $f$ has an asymptote as $x \to \infty$ when there exist numbers $m, b \in \R$ such that
	$$	
		\lim_{x \to \infty} \left[ f(x) - \left( mx + b \right) \right] \; = \; 0.
	$$
Notice that this includes both slant asymptotes (when $m \neq 0$) and horizontal asymptotes (when $m =0$).
	
Consider the following two claims:	
			\begin{center}
				{\bf Claim A:} \quad \quad
					IF $f$ has an asymptote as $x \to \infty$,  \quad
					THEN \DS{\lim_{x \to \infty} \frac{f(x)}{x}} exists.
				
				{\bf Claim B:} \quad \quad 		
					IF \DS{\lim_{x \to \infty} \frac{f(x)}{x}} exists, \quad
					THEN $f$ has an asymptote as $x \to \infty$.
			\end{center}
	\begin{enumerate}
		\item Prove that Claim A is true.
		\item Prove that Claim B is false.

		\item  Here is one more false claim and a bad proof.
			\begin{quotation}
				\noindent
				{\bf Claim C:} Assume the function $f$ is differentiable and that \DS{\lim_{x \to \infty} f(x) = \infty}.
				
				$$  \lim_{x \to \infty} \frac{f(x)}{x} \mbox{ exists} \quad \iff \quad \lim_{x \to \infty} f'(x) \mbox{ exists } $$
				
				
				\noindent
				{\bf ``Proof":}  We can use L'H\^{o}pital's Rule:
					$$
						\lim_{x \to \infty} \frac{f(x)}{x} \; = \; \lim_{x \to \infty} \frac{\frac{d}{dx} f(x)}{\frac{d}{dx} x} 
							\; = \; \lim_{x \to \infty} \frac{f'(x)}{1} \; = \; \lim_{x \to \infty} f'(x)
					$$
					\ \hfill $\square$
			\end{quotation}
			Explain the  error in the proof.
			
			Then prove that the claim is false with a counterexample.
	\end{enumerate}

\end{enumerate}


\end{document}

%%%%%%%%%%%%%%%%%%%%%%%%%%%%%%%%%%%%%%%
%%%%%%%%%%%%%%%%%%%%%%%%%%%%%%%%%%%%%%%
%%%%%%%%%%%%%%%%%%%%%%%%%%%%%%%%%%%%%%%


	
