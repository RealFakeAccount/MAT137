\documentclass[12pt]{exam}
\usepackage{amsmath}
\usepackage{amssymb}
\usepackage{graphicx}
\usepackage{enumitem}
\usepackage{amsfonts}
\usepackage{amssymb}
\usepackage{ifthen}
\usepackage{geometry}
\noprintanswers

\usepackage{tikz}
\usetikzlibrary{shapes,backgrounds}

\usepackage{framed}

\addtolength{\textheight}{3cm}
\addtolength{\topmargin}{-0.8cm}
\addtolength{\textwidth}{2.5cm}
\addtolength{\oddsidemargin}{-1.5cm}
\addtolength{\evensidemargin}{-1.5cm}
\setlength\parindent{0pt}
\setlength{\parskip}{1em}

\newcommand {\DS} [1] {${\displaystyle #1}$}
\newcommand{\vv}{\vspace{.1cm}}

\newcommand{\R}{\mathbb{R}}
\newcommand{\Q}{\mathbb{Q}}
\newcommand{\Z}{\mathbb{Z}}
\newcommand{\N}{\mathbb{N}}

\pagestyle{empty}


%============================================
%137 COLOUR PALETTE
%============================================

\definecolor{137cp1}{RGB}{13, 33, 161}
\definecolor{137cp2}{RGB}{51, 161, 253}
\definecolor{137cp3}{RGB}{255, 67, 101}
\definecolor{137cp4}{RGB}{232, 144, 5}

% to use colours easily
\newcommand{\azul}[1]{{\color{blue} #1}}
\newcommand{\rojo}[1]{{\color{red} #1}}
 
% box in red and blue in math and outside of math
\newcommand{\cajar}[1]{\boxed{\mbox{\rojo{ #1}}}}
\newcommand{\majar}[1]{\boxed{\rojo{ #1}}}
\newcommand{\cajab}[1]{\boxed{\mbox{\azul{ #1}}}}
\newcommand{\majab}[1]{\boxed{\azul{ #1}}}

%============================================
%HYPERLINKS
%============================================

\usepackage{hyperref}
\hypersetup{colorlinks}
\hypersetup{urlcolor=137cp3, linkcolor=137cp1}

%============================================
%Commands used only for this file
%============================================
 
 \usepackage{multicol}
 
%%%%%%%%%%%%%%%%%%%%%%%%%%%%%%%%%%%%%%%%%


\begin{document}

{\large
	\begin{center}
		{\bf MAT 137Y: Calculus with proofs}\\
		{\bf Assignment 7} \\
		{\bf Due on Thursday, February 25 by 11:59pm via Crowdmark}
	\end{center}
}

\begin{quotation}
{\bf Instructions:}
	\begin{itemize}
		\item	 You will need to submit your solutions electronically via Crowdmark.   \href{https://www.math.toronto.edu/~alfonso/137/PS/137_CM.html}{See MAT137 Crowdmark help page for instructions}.  Make sure you understand how to submit and that you try the system ahead of time.  If you leave it for the last minute and you run into technical problems, you will be late.  There are no extensions for any reason.
		\item You may submit individually or as a team of two students.  See the link above for more details.
		\item  You will need to submit your answer to each question separately.
		\item  This problem set is about Units 9 and 10.
	\end{itemize}
\end{quotation}

\begin{enumerate}

\item For every natural number $n$, we define the function $F_n$ by the equation
	\begin{equation} \label{eq:Fn}
		F_n(x) = \int_0^x t^n e^t dt.
	\end{equation}
	\begin{enumerate}
		\item \label{qu:parts} Use integration by parts to write $F_{n}$ in terms of $F_{n-1}$ for $n \geq 1$.  
				
		\item \label{qu:lemma} Prove the following theorem by induction, using your result from Question \ref{qu:parts}:
			\begin{quotation}
				\noindent
				{\bf Theorem.}  For every natural number $n$ there exists a polynomial $P_n$ and a real number $\lambda_n$ such that										\begin{equation*}
					\forall x \in \R, \quad F_n(x) \;  = \; e^x P_n(x)  \; + \; \lambda_n
				\end{equation*}  
			\end{quotation}
		\item   Find (and prove) an explicit formula for $\lambda_n$, as defined in Question \ref{qu:lemma}.
		
			\emph{Hint:}  First find $\lambda_0$ by direct calculation.  Then use the previous questions.
	\end{enumerate}

\vv

\item  Use substitutions to write the following integrals in terms of the functions $F_n$ (as defined by Equation \eqref{eq:Fn}):

	\begin{enumerate}
	\begin{multicols}{2}
		\item \DS{\int_1^x t^p \, e^{at} \, dt}
		\item  \DS{\int x^{2p+1} \, e^{-x^2} \, dx}
		\item \DS{\int_1^x t^{p} \left( \ln t \right)^{q} dt}, \quad for $x>0$
		\item \DS{\int \left( \sin^p x  \right) \left( \cos^3 x \right)  e^{\sin x} \, dx}
	\end{multicols}
	\end{enumerate}
where $p, q \in \N$, $a \in \R$, $a \neq 0$.

\vv

\begin{enumerate}
    \item \emph{Proof:}
		
		Let $p, q\in\N$, $a\in\R$ and $a \neq 0$.
		
		Let $u=at$, so that $du=adt$ since $a$ is a constant. When $t=1$, it can imply that $u=a$. When $t=x$, it can imply that $u=ax$. By substitution of $u$, then we have
		\begin{align*}
		    \int_1^x t^p\ e^{at}\ dt&=\int_a^{ax} (\frac{u}{a})^p\ e^{u}\ \frac{1}{a}du\qquad(\mbox{By Substitution and }\frac{u}{a}=t)\\
		    &=\frac{1}{a^{p+1}}\cdot\int_a^{ax} u^p\ e^{u}\ du\qquad(\mbox{Property of Integral})\\
		    &=\frac{1}{a^{p+1}}\cdot(\int_0^{ax} u^p\ e^{u}\ du\ +\int_a^{0} u^p\ e^{u}\ du)\qquad(\mbox{Property of Integral})\\
		    &=\frac{1}{a^{p+1}}\cdot(\int_0^{ax} u^p\ e^{u}\ du\ -\int_0^{a} u^p\ e^{u}\ du)\qquad(\mbox{Property of Integral})\\
		    &=\frac{1}{a^{p+1}}\cdot(F_p(ax) - F_p(a))\qquad(\mbox{Definition of Equation \eqref{eq:Fn}})\qquad\blacksquare
		\end{align*}
		
		\newpage
    \item \emph{Proof:}
    
        Let $p, q\in\N$, $a\in\R$ and $a \neq 0$.
        
        Let $t=-x^2$, so that $x^2=-t$ and $-2x\ dx=1\ dt$ which implies $x\ dx=\frac{-1}{2}dt$. By substitution of $t$, then we have
        
        \begin{align*}
            \int x^{2p+1} \ e^{-x^2} \ dx &= \int (x^2)^p \ e^{-x^2}\cdot x\ dx\qquad(x^{2p+1}=x^{2p}\cdot x=(x^2)^p\cdot x)\\
            &=\int (-t)^p \ e^t\cdot \frac{-1}{2}dt\qquad(\mbox{By Substitution})\\
            &=(-1)\frac{-1}{2}\int t^p \ e^t\ dt\qquad(\mbox{Property of Integral})\\
            &=\frac{1}{2}\int t^p \ e^t\ dt\\
            &=\frac{1}{2}F_p(x)+C\qquad(\mbox{Definition of Equation \eqref{eq:Fn}})
        \end{align*}
        Since this is an Indefinite Integral, $C$ represents a constant in Indefinite Integral.$\qquad\blacksquare$
        
        \newpage
    \item \emph{Proof:}
        
        Let $p, q\in\N$, $a\in\R$ and $a \neq 0$.
        
        Let $u=\ln t$, so that $du=\frac{1}{t}dt$ which imply $t\ du=dt$ and also we can get:
        \begin{align*}
            u&=\ln t\\
            e^u&=e^{\ln t}\\
            e^u&=t
        \end{align*}
        Then $t\ du=dt\iff e^u\ du=dt$. When $t=1$, it can imply that $u=\ln 1=0$. When $t=x$, it can imply that $u=\ln x$.
        
        Let $x>0$. By substitution of $u$, then we have
        \begin{align*}
            \int_1^x t^{p} \left( \ln t \right)^{q} dt&=\int_0^{\ln x} (e^u)^{p}u^{q}e^u du\qquad(\mbox{By Substitution and }e^u=t)\\
            &=\int_0^{\ln x} e^{u(p+1)}u^{q}du
        \end{align*}
        Let $u(p+1)=k$, so that $dk=(p+1)du$ since $(p+1)$ is a constant. When $u=0$, it can imply that $k=0\cdot(p+1)=0$. When $u=\ln x$, it can imply that $k=(p+1)\ln x$. By substitution of $k$, then we have
        \begin{align*}
            \int_0^{\ln x} e^{u(p+1)}u^{q}du&=\int_0^{(p+1)\ln x} e^k(\frac{k}{p+1})^{q}\frac{1}{p+1}dk\quad(\mbox{By Substitution and $u=\frac{k}{p+1}$ and $du=\frac{1}{p+1}dk$})\\
            &=(\frac{1}{p+1})^{q+1}\int_0^{(p+1)\ln x} e^kk^{q}dk\qquad(\mbox{Property of Integral})\\
            &=(\frac{1}{p+1})^{q+1}F_q((p+1)\ln x)\qquad(\mbox{Definition of Equation \eqref{eq:Fn}})\qquad\blacksquare
        \end{align*}
        
        \newpage
    \item \emph{Proof:}
    
        Let $p, q\in\N$, $a\in\R$ and $a \neq 0$.
        
        Let $t=\sin x$, so that $dt=\cos x\ dx$ which imply $dx=\frac{1}{\cos x}dt$. By trigonometry property, we know that $\sin^2 x+\cos^2 x=1$, then $t^2+\cos^2 x=1$, then $\cos^2 x=1-t^2$.
        
        By substitution of $t$, then we have
        \begin{align*}
            \int (\sin^p x) ( \cos^3 x)  e^{\sin x}\ dx&=\int (\sin^p x) \cos^2 x\ e^{\sin x}\ \cos x\ dx\qquad(\cos^3 x=\cos^2 x\cdot\cos x)\\
            &=\int t^p (1 - t^2)\ e^t\ dt\qquad(\mbox{By Substitution})\\
            &=\int t^p\ e^t\ dt\ - \int t^p\cdot t^2\ e^t\ dt\qquad(\mbox{Property of Integral})\\
            &=\int t^p\ e^t\ dt\ - \int t^{p+2}\ e^t\ dt\\
            &=F_p(x)-F_{p+2}(x)+C\qquad(\mbox{Definition of Equation \eqref{eq:Fn}})
        \end{align*}
        
        Since this is an Indefinite Integral, $C$ represents a constant in Indefinite Integral.$\qquad\blacksquare$
        
        \newpage
\end{enumerate}

\item Before you attempt this question, work on the Practice Problems for Unit 10 (specifically the sections on Mass Density and Center of Mass).  Otherwise the question won't make sense.

Every time we have two point masses in a closed space, they generate something called ``macguffin".  If we have a mass $m_1$ at point $P_1$ and a mass $m_2$ at point $P_2$, then they generate a macguffin with value $G$ given by
	$$
		G =  m_1 m_2 z^2 
	$$
where $z$ is the distance between $P_1$ and $P_2$.

If we have more than two masses, every pair of masses generates a macguffin.  For example, if we have three masses (call them 1, 2, and 3) at three different points, then the total macguffin generated by them is the sum of 
	\begin{itemize}
		\item the macguffin generated by masses 1 and 2, 
		\item the macguffin generated by masses 1 and 3,
		\item the macguffin generated by masses 2 and 3.
	\end{itemize}
\begin{enumerate}
	\item  Assume we have $N$ masses on $N$ different positions on the $x$-axis: a mass $m_1$ at $x_1$, a mass $m_2$ at $x_2$, ..., a mass $m_N$ at $x_N$.  Obtain a formula for the total macguffin generated by the masses using sigma notation. 
	\item  Assume that instead of a collection of point masses we have continuous masses (which is more realistic).  Specifically, we have a bar on the $x$-axis, from $x=a$ to $x=b$, whose mass density at the point $x$ is given by $\mu(x)$.  Assume $\mu$ is a continuous function.  Obtain a formula for the total macguffin generated by the bar using integrals.
\end{enumerate}



\end{enumerate}
\end{document}
%%%%%%%%%%%%%%%%%%%%%%%%%%%%%%%%
%%%%%%%%%%%%%%%%%%%%%%%%%%%%%%%%




	
